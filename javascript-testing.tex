% !TEX TS-program = xelatex
% !TEX encoding = UTF-8

\documentclass[10pt, a4paper]{article}

\usepackage{xltxtra} % loads fontspec and xunicode

\defaultfontfeatures{Mapping = tex-text, Scale = MatchLowercase}
\setmainfont{Adobe Garamond Pro}
\setsansfont{Helvetica}
\setmonofont{Bitstream Vera Sans Mono}

\usepackage{polyglossia}
\setdefaultlanguage{german}

\usepackage{minted}

\setlength\parindent{0mm}
\setlength\parskip{2mm}

\usepackage{url}

\usepackage{color}

\usepackage[pdfborder={0 0 0 0}]{hyperref}

\title{JavaScript Testing}
\author{Peter Krenn}

\begin{document}

\maketitle

\begin{abstract}
Zusammenfassung
\end{abstract}

\tableofcontents

\section{Einleitung}

\section{QUnit}

\emph{QUnit}\cite{zaefferer_qunit_2011} ist Teil des \emph{jQuery}
Projektes\cite{resig_jquery_2011} und wird bei diesem auch als \emph{test
framework} eingesetzt. Es ist einem Design nachempfunden, dass ursprünglich als
\emph{SUnit}\cite{beck_simple_1994} für \emph{Smalltalk} entwickelt wurde und als \emph{jUnit} für
\emph{Java} große Verbreitung fand. Allgemein nennt man diese Gruppe
\emph{xUnit test frameworks}\cite{fowler_xunit_2010}.

\begin{minted}[gobble = 2]{ruby}
  class Asdf < Peter
    def asdf(asdf)
      13 + 12
    end
  end
\end{minted}

\begin{flushleft}
  \bibliography{javascript-testing}
  \bibliographystyle{plain}
\end{flushleft}

\end{document}
